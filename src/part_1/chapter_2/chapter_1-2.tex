\section{The robust counterpart}

We study the LO problem
\begin{align}\label{p1c2:Loproblem}
	\min\limits_x\{c^Tx\mid Ax\leq b\},
\end{align}
where $c\in\reals^n$, $b\in\reals^m$ and $A\in\reals^m\times\reals^n$. We address the problem where $(c,A,b)$ are uncertain, but known to reside in a compact, specified uncertainty set $\uncset$. In this case, problem \eqref{p1c2:Loproblem} is not a single deterministic problem, but a family of problems
\begin{align}
	\min\limits_x\{c^Tx\mid Ax\leq b\}_{(c,A,b)\in\uncset}.
\end{align}
We assume that all decision variables must be determined before the uncertain parameters become known, and that the problem constraints are hard, i.e. they must be satisfied for any uncertain parameters within $\uncset$. The RO approach converts this family of problems into one single deterministic problem, referred to as \textit{robust counterpart}:
\begin{align}\label{p1c2:robustcounterpart}
	\pi* = \min\limits_x \{t \mid Ax\leq b,\;c^Tx\leq t,\;\;\forall (c,A,b)\in\uncset\}.
\end{align}
A vector $x*$ satisfying feasibility for all $(c,A,b)\in\uncset$ and with an objective of at most $\pi*$ is a \textit{robust optimal solution}. It is often said that this solution optimizes for the worst case, i.e., that
\begin{align*}\label{p1c2:minmax}
\pi* = \min\limits_x \max\limits_{(c,A,b)\in\uncset} \{c^T x:\;Ax\leq b\}.
\end{align*}
However, \eqref{p1c2:robustcounterpart} may have different worst-case scenarios depending on which constraint is considered, while \eqref{p1c2:minmax} has one worst-case scenario, namely the one defining the optimal objective value. In addition, in \eqref{p1c2:robustcounterpart}, a feasible solution $x$ must subject to all scenarios in $\uncset$, while in \eqref{p1c2:minmax}, the scenarios are subject to the decision $x$ (i.e., we only consider scenarios for which $x$ is feasible). The statement is thus only true in the sense that non-worst case scenarios do not influence the problem. 

Note, that in \eqref{p1c2:robustcounterpart}, all of the uncertainty, including $c$, takes place in the constraints. We may therefore assume the following generic form
\begin{align*}
	\min\limits_x & c^Tx \\
	\text{s.t.} & a_i^Tx-b_i\leq 0, \quad \forall (A,b)\in\uncset,\quad\forall i\in[m],
\end{align*}
where $[m]=\{1,\dots,m\}$, $c$ is certain, and $\{a_i,b_i\}_{i\in[m]}$ are uncertain. If $\uncset$ is not a finite set, this problem has an infinite number of constraints. We observe that $\uncset$ may equivalently be replaced by its convex hull: Suppose that $a_1^Tx-b_1\leq 0$, as well as $a_2^tx-b_2\leq 0$ for $a_1,a_2\in\uncset$. Then, for $\lambda (a_1,b_1) + (1-\lambda)(a_2,b_2)\in conv(\uncset)$, where $\lambda\in(0,1$, we have
\begin{align*}
(\lambda a_1 + (1-\lambda)a_2 )^Tx - \lambda b_1 + (1-\lambda)b_2 = \lambda (a_1^Tx-b_1) + (1-\lambda)(a_2^T-b_2)\leq 0. 
\end{align*} 
We will now derive a general form for the uncertain constraints which facilitates our studies. Firstly, we note that we may disassemble the uncertainty set row-wise without modifying the robustness, i.e. reformulate it such that any uncertain parameter only appears in one constraint of the uncertainty set. For instance, consider the simple uncertain constraints
\begin{align*}
x_i-b_i \leq 0,\; i=1,2.
\end{align*}
The feasible values for $x$ remain the same, regardless of whether $b\in \uncset =\{(b_1,b_2)^T\mid b_1,b_2\geq0,\;b_1+b_2\leq 1\}$ or simply $b_i\in[0,1]$ for $i=1,2$, respectively.
Thus, we can study the impact of uncertainty on each constraint individually. We will in the following drop the constraint index $i$, obtaining constraints of the form
\begin{align*}
	a^Tx-b\leq0,\quad\forall (a,b)\in\uncset,
\end{align*}
where $a\in\reals^n,\; b\in\reals$, and $\uncset$ stands for $\uncset_i$. Furthermore, we can assume without loss of generality, that the right-hand side $b$ is certain - if not, we can introduce an extra variable $x_{n+1}$ and change the constraint to
\begin{align*}
	& a^Tx-bx_{n+1}\leq0,\quad\forall (a,b)\in\uncset,\\
	& x_{n+1} = 1.
\end{align*}
Now, we consider the uncertain parameter $a$, separating it into a fixed nominal value $\ol{a}\in\reals^n$, common to all scenarios, and the key uncertainty $\keyunc\subseteq\reals^L$. Since the uncertain parameters may be correlated among each other, we model
\begin{align*}
a = \ol{a} + Pz, z\in\keyunc,
\end{align*}
where $P\in\reals^{n\times L}$. Note, that since the uncertainty is row-wise, we may find an equivalent uncertainty set such that $P$ is a diagonal matrix, but this may not always be to the benefit of clarity. Thus ,we obtain the general constraint
\begin{align*}
(\ol{a}+Pz)^Tx\leq b, \quad \forall z\in\keyunc.
\end{align*}

\section{Fundamental Uncertainty Sets}

For uncertainty sets with specific fundamental shapes, it is possible to derive tractable robust counterparts.

\subsection{Box uncertainty set}

The uncertanity set
\begin{align*}
	\keyunc = \{z\mid -\rho_i\leq z_i\leq\rho_i, \forall i\in[L]\} = \{z\mid ||z||_\infty\leq\rho\}
\end{align*}
has the shape of  a box, where every component can independently take values in the interval $[-\rho,rho]$. This models situations in which we only have informations about the boundaries of the uncertainty, but not the relationship between the different parameters.
We derive a tractable reformulation as follows:

\begin{align*}
 & (\ol{a}+Pz)*Tx\leq b, \quad\forall z: \norm{z}_\infty\leq\rho \\
 \ifff & \ol{a}^Tx+\max\limits_{z:\norm{z}_\infty\leq\rho} (P^Tx)^Tz\leq b \\
 \ifff & \ol{a}^Tx+\rho\norm{P^Tx}_1 \leq b.
\end{align*}
This constraint can be modeled as a set of linear constraints, leading to a linear, but slightly larger RC where the uncertanity has been replaced by the safety term $\rho\norm{P^Tx}_1$.

\subsection{Polyhedral uncertainty set}

A polyhedral uncertainty set is specified by $q$ linear inequalities:
\begin{align*}
& (\ol{a}+Pz)*Tx\leq b, \quad\forall z: d-Dz\geq 0, \\
\ifff & \ol{a}^Tx + \max\limits_{z:d-Dz\geq 0} (P^Tx)^Tz\leq b \\
\ifff & \ol{a}^Tx + \min\limits_{y\geq 0:D^Ty=P^Tx} d^Ty \leq b.
\end{align*}
where $d\in\reals^q$ and $D\in\reals^q\times L$. If, for some $y\geq0$ such that $D^Ty = P^Tx$, we have $\ol{a}^Tx+d^Ty\leq b$, then this certainly also holds for the corresponding minimum. We can therefore drop the $\min$-operator, obtaining the equivalent conditions  
\begin{align*}
 \ol{a}^Tx + d^Ty \leq b,\;D^Ty=P^Tx,\;y\geq 0.
\end{align*}
As a system of linear constraints, the corresponding RC is tractable, albeit of increased size compared to the original problem. Since box uncertainty is a special case of polyhedral uncertainty, this is a generalization of the box reformulation.

\subsection{Budget uncertainty set}

In practice, it is usually unlikely that extreme senarios like the corners of box uncertainty sets will occur, where all parameters show maximal deviation from the nominal value. To make the RO problems less conservative, we may in addition to the box constraints use so-called budget constraints on the uncertainty, which ensure that the aggregated sum of deviations $z_i$ may not exceed the uncertainty budget $\Gamma$.
\begin{align}\label{p1c1:budgetunc}
\uncset = \{z:\;\norm{z}_infty\leq\rho,\;\norm{z}_1\leq\Gamma\}
\end{align}
If $\Gamma$ is small enough, this set cuts off the corners of the box uncertainty set. As a subset of the box uncertainty set, the RC for budget uncertainty is thus at least as tight as the box RC. By writing \eqref{p1c1:budgetunc} in the format of $d-Dz\geq 0$, we can show that $x$ satifies
\begin{align*}
(\ol{a}+Pz)^Tx\leq b,\quad \forall z:\;\norm{z}_infty\leq\rho,\;\norm{z}_1\leq\Gamma,
\end{align*}
if and only if there exists $y$ such that $(x,y)$ satisfies
\begin{align*}
\ol{a}^Tx+\rho\norm{y}_1+\Gamma\norm{P^Tx-y}_\infty\leq b.
\end{align*}

As a final remark, the given derivations also hold for integer $x$, since the duality used in the derivation was with respect to $z$, and not $x$.