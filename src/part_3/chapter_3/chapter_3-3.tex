RO uses optimization to computationally solve optimization problems under uncertainty in very high dimensions, and it uses probability theory to select uncertainty sets that give strong guarantees the optimal robust solutions will be feasible and near optimal \highlight{if the uncertain parameters take values from arbitrary distributions}{why this addition?}. In this chapter, we learn how to select uncertainty sets based on the available data.

\section{Normally Distributed Uncertain Parameters}

In some cases we are able to conclude that the uncertain parameter $\tilde{z}$ follows a normal distribution, for example from historical data, or due to the central limit theorem. Suppose that the distribution has mean $\mu$ and covariance matrix $E$. Then $\tilde{w}=(Px)^T\tilde{z}$ has expected value $\ev [\tilde{w}] = (Px)^T\mu$ and $\var (\tilde{w}) = x^TP^TE Px$. Suppose that we want a constraint to hold with probability $1-\epsilon$, i.e.,
\begin{align*}
\prob ( (\ol{a}+P\tilde{z})^Tx \leq b) \geq 1-\epsilon.
\end{align*}
Due to the normal distribution, this is satisfied if and only if
\begin{align*}
(\ol{a}+P\mu)^Tx+\rho_{1-\epsilon}\sqrt{x^TP^TE Px}\leq b,
\end{align*}
where $\rho_{1-\epsilon}$ is the $(1-\epsilon)$-percentile of a standard normal distribution. This is equivalent to $(\ol{a}+Pz)^Tx\leq b$ for the uncertainty set 
\begin{align*}
\keyunc = \{z\; :\;\norm{E^{-\frac{1}{2}}(z-\mu)}\leq\rho_{1-\epsilon}\},
\end{align*}
which is ellipsoidal. Thus, ellipsoidal uncertainty sets provide probabilistic guarantees for normally distributed uncertain parameters. Note, that we want the constraint to hold with probability $1-\epsilon$, which is a weaker statement than $\prob(\tilde{z}\in\keyunc )\geq 1-\epsilon$.

\section{Selecting Parameters for Uncertainty Sets}\label{p3c3sec:param}

Consider the ball uncertainty set $\keyunc_\rho = \{z\in\reals^L:\;\norm{z}_2\leq\rho\}$ and the robust constraint
\begin{align*}
(\ol{a}+Pz)^Tx\leq b,\quad \forall z\in\keyunc_\rho,
\end{align*}
with corresponding RC
\begin{align}\label{p3c3:RCcon}
\ol{a}^Tx + \rho\norm{P^Tx}_2\leq b.
\end{align}
We now wish to select $\rho$ such that the probability of the constraint being satisfied is very high, when $\tilde{z}$ is stochastic.

\begin{theorem}
If $\tilde{z}_i, i\in L$ are independent random variables with support in $[-1,1]$ and zero mean, and $x$ satisfies \eqref{p3c3:RCcon}, then
\[\prob ( (\ol{a}+P\tilde{z})^Tx\leq b)\geq 1-\exp (-\rho^2/2).\]
\end{theorem}
\highlight{Intuitively, this is because the weighted sum $X(\tilde{\xi})= \sum_{i=1}^k w_i\tilde{\xi}_i$ of a collection of zero-mean variables $\tilde{\xi}$ is more likely to be zero than the individual variables, since they may cancel each other out.}{Is this true?}

By choosing $\rho = \sqrt{2\ln(1/\epsilon)}$, we obtain $\prob ( (\ol{a}+P\tilde{z})^Tx\leq b)\geq 1-\epsilon$. This gives us the opportunity to calculate the values of $\rho$, the uncertainty set radius, as a function of $\epsilon$, the violation probability. Most remarkably, these values of $\rho$ are independent of the dimension $L$ of the uncertainty set.

\begin{figure}
\begin{tabular}{c|c|c|c|c|c|c|c}
 $\epsilon$	& $10^{-12}$	& $10^{-6}$	& $10^{-4}$ & $10^{-3}$	& $0.005$	& $10^{-2}$	& $10^{-1}$ \\
 $\rho$ 			&7.44					& 5.25			& 4.29		   & 3.71				& 3.26		& 3.03			& 2.14 \\
\end{tabular}
\end{figure}

A similar result is obtainable for the budget uncertainty set
\begin{align*}
\keyunc_\Gamma = \{z\in \reals^L\; : \; \norm{z}_1\leq \Gamma, \norm{z}_\infty \leq 1\}
\end{align*}
with the RC constraint
\begin{align}\label{p3c3:gammacon}
\ol{a}^Tx + \Gamma\norm{P^Tx-y}_\infty + \norm{y}_1 \leq b.
\end{align}
\begin{theorem}
If $\tilde{z}_i, i\in L$ are independent random variables with support in $[-1,1]$ and zero mean, and $x$ satisfies \eqref{p3c3:gammacon}, then
\[\prob ( (\ol{a}+P\tilde{z})^Tx\leq b)\geq 1-\exp (-\Gamma^2/(2L)). \]
\end{theorem}
Choosing $\Gamma = \sqrt{2\ln (1/\epsilon)}\sqrt{L}$, then, again, we can reach a confidence level of $1-\epsilon$ that the constraint will be satisfied. However, in this case, $\Gamma$ does depend on the dimension $L$ of the uncertainty set .
\section{Comparing Uncertainty Sets}

In this section, we consider an example in single-period portfolio optimization to compare different choices of uncertainty sets. There are 200 assets, the last of which is leaving the money in the bank with a certain annual return of $r_{200} =1.05$. All other assets have uncertain annual returns $r_i, i=1,\dots,199$, modeled by independent random variables taking values in $[\mu_i-\sigma_i, \mu_i+\sigma_i]$ with expected values $\mu_i$. We set
\begin{align*}
\mu_i &= 1.05 + 0.3\frac{200-i}{199}, \\
\sigma_i &= 0.05 + 0.6\frac{200-i}{199},
\end{align*}
for $i=1,\dots,199$, so that for higher $i$, we have lower expected return and lower variability. The objective is to distribute a symbolic $1\$$ between the assets in order to maximize the expected return. The uncertain LO problem is thus
\begin{align*}
\max\limits_{x,t} & t \\
\stnew& \sum\limits_{i=1}^199 r_ix_i + r_{200}x_{200}\geq t,\\
& \sum\limits_{i=1}^{200} x_i = 1, \\
& x\geq 0,
\end{align*}
where $x_i$ is the capital invested into asset $i$, and the uncertain data  are the returns $r_i = \mu_i + \sigma_iz_i)x$, $i\in[199]$, where $z_i$ are independent random variables with zero mean varying in the interval $[-1,1]$. The resulting RC is
\begin{align*}
\max\limits_{x,t} & t \\
\stnew & \sum\limits_{i=1}^199 (\mu_i+\sigma_iz_i)x_i +1.05x_{200}\geq t,\quad \forall z\in\keyunc,\\
& \sum\limits_{i=1}^{200} x_i = 1, \\
& x\geq 0,
\end{align*}
for a variety of uncertainty sets $\keyunc$. We want the expected return to be guaranteed with a probability of at least $99.5\%$. Using the results from Section \ref{p3c3sec:param}, we select  values $\rho$ and $\Gamma$ for the uncertainty sets that give us the probabilistic guarantees we desire. We thus consider the following sets $\keyunc$:
\begin{itemize}
	\item[Box:] $\keyunc_1 = \{z:\norm{z}_\infty \leq 1\}$, which models only the fact that the uncertainties take values in [-1,1].
	\item[Ball-Box:] $\keyunc_2 = \{z:\norm{z}_\infty \leq 1, \norm{z}_2\leq \rho\}$, with $\rho = \sqrt{2 \ln (1/\epsilon)}=3.26$, ensuring that the uncertain constraint is satisfied with probability at least $1-\epsilon = 0.995$.
	\item[Budget:] $\keyunc_3 = \{z:\norm{z}_\infty \leq 1, \norm{z}_1\leq \Gamma\}$, where $\Gamma = 45.92$, again ensuring constraint satisfaction with probability $1-\epsilon = 0.995$. 
\end{itemize}
For the three uncertainty sets, we get the optimal guaranteed returns of $1.05$ (Box), $1.12$ (Ball-Box) and $1.10$ (Budget). \highlight{In this case, the ball-box uncertainty set has an advantage over the budget uncertainty set, since the choice of parameter $\rho$ is independent of the dimension of the uncertainty set. $\Gamma$ depends on this dimension, which makes the uncertainty set larger.}{good insight}